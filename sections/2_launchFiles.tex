% Section 2 - Launch Files
% Roberto Masocco <roberto.masocco@uniroma2.it>
% May 1, 2022

% ### Launch Files ###
\section{Launch Files}

% --- Scripting ROS 2 Architectures ---
\begin{frame}{Scripting ROS 2 Architectures}
A ROS 2-based control architecture for a robot can easily get to have 20 nodes or more.\\\vspace{10pt}
It then becomes critical to be able to \textbg{automate startup and configuration} of all the modules, or some subsets, also for testing.\\\vspace{10pt}
That is what the ROS 2 \textbg{Launch System} is for.
\end{frame}

% --- Launch Files ---
\begin{frame}{Launch Files}
\textbg{Launch files} are \textbg{Python scripts} that specify how ROS 2 modules must be \textbg{located}, \textbg{configured} and \textbg{started}. Their format is such that the Launch System can parse and integrate them when invoked.\\\vspace{10pt}
Many things can be configured about modules in such files:
\begin{itemize}
  \item console and text files \textbg{logs};
  \item command line \textbg{arguments};
  \item node \textbg{parameters};
  \item \textbg{remappings}.
\end{itemize}
\vspace{10pt}
Targeted modules may even not be ROS 2 nodes.\\\vspace{10pt}
Launch files may be \textbg{included}, so that large architectures can be started with single commands.
\end{frame}
\begin{frame}[fragile]{Launch Files}
\begin{columns}\column{.9\textwidth}
\begin{lstlisting}[language=Python, caption=Minimal example of a launch file]
from launch import LaunchDescription
from launch_ros.actions import Node

# The following function MUST be specified

"""Builds a launch description."""
def generate_launch_description():
  ld = LaunchDescription()
  node = Node(
    package='PACKAGE_NAME',
    executable='EXECUTABLE_NAME'
  )
  ld.add_action(node)
  return ld
\end{lstlisting}
\end{columns}
\end{frame}

% --- Coding Launch Files
\begin{frame}{Coding Launch Files}
\begin{block}{Launch Files Best Practices}
  \begin{itemize}
    \item Their extension is usually \texttt{.launch.py}.
    \item They are usually placed in a directory named \texttt{launch} that is installed in the workspace path during build.
    \item A module can have its own launch files but those for the entire architecture must form an appropriate package, whose name is usually \texttt{PROJECT\_bringup}.
  \end{itemize}
\end{block}
A comprehensive description of all the features of launch files can be found in \href{https://github.com/IntelligentSystemsLabUTV/ros2-examples/blob/galactic/launch_files.md}{\color{blue}\underline{launch\_files.md}}.
\end{frame}

% --- Example: Bringup Package ---
\begin{frame}{Example: Bringup Package}
  Now go have a look at the \href{https://github.com/IntelligentSystemsLabUTV/ros2-examples/tree/galactic/src/ros2_examples_bringup}{\color{blue}\underline{ros2-examples/src/ros2\_examples\_bringup}} package!
\end{frame}

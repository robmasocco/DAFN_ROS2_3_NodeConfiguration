% ### SECTION NAME ###
\section{Section Name}
\graphicspath{{figs/section1/}} % Use this to include figures easily

% --- Frame: text ---
\begin{frame}{Frame Title}{Frame Subtitle}
    Frame contents
     \begin{itemize}
        \item Item 1
        \item Item 2
        \begin{itemize}
            \item Subitem 2.1 
            \item Subitem 2.2 
        \end{itemize}
    \end{itemize}
    This is \textbg{bold text for normal text}.
\end{frame}

% --- Frame: blocks ---
\begin{frame}{Frame Title}{Frame Subtitle}
    \begin{block}{Block}
        This is \textbf{bold text for blocks}.
    \end{block}
    \begin{alertblock}{Alert Block}
        This is \textbr{bold text for alert blocks}.
    \end{alertblock}
    \begin{example}
        This is \textbf{bold text for example blocks}.
    \end{example}
    \begin{block}{}
        \centering
        This is \textbf{bold text for unnamed blocks}.
    \end{block}
\end{frame}

% --- Frame: equation ---
\begin{frame}{Frame Title}{Frame Subtitle}
\begin{block}{Numbered equation}
    \begin{equation}
        \label{eq:nlsys}
        \left\{
        \begin{aligned}
            \dot{x} &= f(x, u)\\
            y &= h(x,u)
        \end{aligned}
        \right.
    \end{equation}
\end{block}
\begin{block}{Unnumbered equation}
    \begin{equation*}
        \label{eq:lsys}
        \left\{
        \begin{aligned}
            \dot{x} &= Ax + Bu\\
            y &= Cx
        \end{aligned}
        \right.
    \end{equation*}
\end{block}
\end{frame}

% --- Frame: TikZ ---
\begin{frame}{Frame Title}{Frame Subtitle}
    \begin{figure}
        \begin{center}
            \begin{tikzpicture}[auto, node distance=1cm,>=latex']
                % Blocks
                \node (controller) [block]  {$Controller$};
                \node (ref) [input, left=of controller.170] {};
                \node (feedback) [input, left=of controller.190] {};
                \node (plant) [block, right=2 of controller] {$Plant$};
                \node (disturbance) [input, above=of plant] {};
                \node (doty) [coordinate, right=of plant] {};
                \node (output) [output, right=of doty] {};
                % Connections
                \draw [->] (ref) -- node {$r$} (controller.170);
                \draw [->] (feedback) -- node [below] {$u_c$} (controller.190);
                \draw [->] (controller) -- node[pos=0.2] {$y_c$} node[pos=0.8] {$u$} (plant);
                \draw [->] (plant) -- node {$y$} (output);
                \draw [->] (disturbance) -- node {$d$} (plant);
                \draw [-] (doty) -- ++ (0,-1.5) -| node {} (feedback);
            \end{tikzpicture}
            \caption{Feedback control system}
            \label{fig:controlscheme}
        \end{center}
    \end{figure}
\end{frame}
